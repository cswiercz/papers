\documentclass{amsart}

\usepackage{amsmath,amssymb}
\usepackage{algorithm}
\usepackage{algpseudocode}

\title{The Siegel Transformation}
\author{
  Chris Swierczewski \\
  Grady Williams
}
\date{11 April 2013}

\begin{document}

\maketitle

%%%%%%%%%%%%%%%%%%%%%%%%%%%%%%%%%%%%%%%%%%%%%%%%%%%%%%%%%%%%%%%%%%%%%%%%%%%%%%%
\section{Siegel Transforms}
%%%%%%%%%%%%%%%%%%%%%%%%%%%%%%%%%%%%%%%%%%%%%%%%%%%%%%%%%%%%%%%%%%%%%%%%%%%%%%%



We wish to compute the Riemann theta function
\[
  \theta(z,\Omega) = \sum_{n \in \mathbb{Z}^g}
  e^{2\pi\i \left( \tfrac{1}{2} n \cdot \Omega n + n \cdot z \right)}
\]
to a desired accuracy $\epsilon$. Let $z = x+iy,\Omega = X + iY$ and $Y = T^TT$
be the Cholesky decomposition of $Y$. Let $\Lambda$ be the lattice
\[
  \Lambda = \sqrt{\pi}T \left( \mathbb{Z}^g + [[Y^{-1}y]] \right)
\]
and $\rho$ be the length of the shortest lattice vector. (Note that
this is the length of the shortest lattice vector of $T\mathbb{Z}$
scaled by $\sqrt{\pi}$.) Then, by [CRTF], to compute
$\theta(z,\Omega)$ with error $\epsilon$ it is sufficient to compute
the ``oscillatory part'' over the integer points
\[
  U_R = \left\{ n \in \mathbb{Z}^g \; | \; \pi n \cdot Y n < R \right\}
\]
where $R$ is the real positive solution to
\[
  \epsilon = g2^{g-1}\Gamma(g/2,(R-\rho/2)^2)/\rho^g
\]
and $\Gamma(s,x)$ is the incomplete gamma function.

Note that $U_R$ is a collection of integer vectors lying inside the
ellipsoid $\pi z \cdot Yz = R$. If the ellipsoid has high eccentricity
then it is possible that both $|U_R|$ is large and the magnitudes of
the elements of $U_R$ are large. These two situations can contribute
to numerical error in an implementation of the algorthm.



%%%%%%%%%%%%%%%%%%%%%%%%%%%%%%%%%%%%%%%%%%%%%%%%%%%%%%%%%%%%%%%%%%%%%%%%%%%%%%%
\subsection{Modular Transformation and Theta Functions}
%%%%%%%%%%%%%%%%%%%%%%%%%%%%%%%%%%%%%%%%%%%%%%%%%%%%%%%%%%%%%%%%%%%%%%%%%%%%%%%



This high eccentricity situation can be mitigated by using modular
transformations on the Riemann matirx $\Omega$ in order to map the
problem into one where we instead compute $\theta(\hat{z},
\hat{\Omega})$ on a new lattice $\hat{U}_R$ that has smaller
eccentricity. Given a symplectic matrix over the integers
\[
\Gamma = \left( \begin{matrix} a & b \\ c & d \end{matrix} \right)
\in SP(2g,\mathbb{Z})
\]
this corresponds to the modular transformation
\[
\Omega \mapsto \hat{\Omega} = (a\Omega + b) (c\Omega + d)^{-1}.
\]
By [Tata I], the Riemann theta function satisfies the following
functional equations under certain modular transformations. The listed
modular tranformations form a basis for the modular group $SP(2g,
\mathbb{Z})$.

\begin{table}[h]
\label{tbl: transformations}

\centering
\begin{tabular}{ccc}
  Symplectic Transform & Transformed Matrix & Functional Eq. \\ \hline
  $\Gamma = \left[ \begin{smallmatrix} a & 0 \\ 0 &
      a^{-T} \end{smallmatrix} \right]$ & $\hat{\Omega} = a \Omega
  a^T$ & $\theta(z,\Omega) = \theta(az,a\Omega a^T)$ \\ $\Gamma =
  \left[ \begin{smallmatrix} I & b \\ 0 & I \end{smallmatrix} \right]$
  & $\hat{\Omega} = \Omega + b$ & $\theta(z,\Omega) = \theta(z, \Omega
  + b)$
\end{tabular}
\end{table}



\section{Algorithm}

\begin{algorithm}
\caption{Compute the Siegel Transform of a Riemann Matrix}
\label{alg: siegel}
\begin{algorithmic}
  \Procedure{Siegel}{$\Omega$} 
  \Comment{$\Omega$ a $g \times g$ Riemann matrix}

  \State $\Gamma \gets I_{2g}$
  \Comment{$I_{2g} = g \times g$ identity matrix}
  \State $a \gets \left[ \begin{smallmatrix} 0 & \mathbf{0}^T \\ \mathbf{0} & I_{g-1} \end{smallmatrix} \right]$
  \Comment{$\mathbf{0} = $ column vector with $g-1$ zeros}
  \State $b \gets \left[ \begin{smallmatrix} -1 & \mathbf{0}^T \\ \mathbf{0} & \mathbf{0}_{g-1} \end{smallmatrix} \right]$
  \Comment{$\mathbf{0}_{g-1} = (g-1) \times (g-1)$ zero matrix}
  \State $c \gets \left[ \begin{smallmatrix} 1 & \mathbf{0}^T \\ \mathbf{0} & \mathbf{0}_{g-1} \end{smallmatrix} \right]$
  \State $d \gets \left[ \begin{smallmatrix} 0 & \mathbf{0}^T \\ \mathbf{0} & I_{g-1} \end{smallmatrix} \right]$
  \State $\texttt{finished} \gets \texttt{False}$
  \While{{\tt finshed} $ \neq $ {\tt True}}
    \State $Y \gets \text{Im } \Omega$
    \State $T \gets \textsc{Cholesky}(Y)^T$
    \State $L \gets \textsc{SortedLLL}(T)$
    \Comment{LLL with cols. sorted by length}
    \State $U \gets$ solution to $TU = L$
    \State $\Omega \gets U^T \Omega U, \quad \Gamma \gets \left[ \begin{smallmatrix} U^T & 0 \\ 0 & U^{-1} \end{smallmatrix} \right] \Gamma$
    \State $X \gets \text{Re } \Omega$
    \State $\Omega \gets \Omega - [X], \quad \Gamma \gets \left[ \begin{smallmatrix} I & -[X] \\ 0 & I \end{smallmatrix} \right] \Gamma$
    \If{$|\Omega_{11}| \geq 1$}
      \State $\texttt{finished} \gets \texttt{True}$
    \Else
      \State $\Gamma \gets \left[ \begin{smallmatrix} a & b \\ c & d \end{smallmatrix} \right] \Gamma$
      \State $\Omega \gets (a\Omega + b)(c \Omega + d)^{-1}$
    \EndIf
  \EndWhile
  \State \textbf{return} $\Omega, \Gamma$
\EndProcedure
\end{algorithmic}
\end{algorithm}




\end{document}

