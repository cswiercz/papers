\documentclass[12]{article}

\usepackage{amsmath,amssymb}
\usepackage{color}
\usepackage{algorithm}
\usepackage{algpseudocode}
\usepackage{hyperref}
\hypersetup{
    bookmarks=true,         % show bookmarks bar?
    unicode=false,          % non-Latin characters in Acrobat’s bookmarks
    pdftoolbar=true,        % show Acrobat’s toolbar?
    pdfmenubar=true,        % show Acrobat’s menu?
    pdffitwindow=false,     % window fit to page when opened
    pdfstartview={FitH},    % fits the width of the page to the window
    pdftitle={Efficient Modular Computation of the Riemann
      Theta Function},    % title
    pdfauthor={Swierczewski and Williams},     % author
    pdfcreator={Swierczewski},   % creator of the document
    pdfkeywords={mathematics} {theta functions} {siegel transform},
    pdfnewwindow=true,      % links in new window
    colorlinks=false,       % false: boxed links; true: colored links
    linkcolor=red,          % color of internal links
    citecolor=green,        % color of links to bibliography
    filecolor=magenta,      % color of file links
    urlcolor=cyan           % color of external links
}

\newtheorem{theorem}{Theorem}
\newtheorem{definition}{Definition}

\DeclareMathOperator{\ZZ}{\mathbb{Z}}
\DeclareMathOperator{\RR}{\mathbb{R}}
\DeclareMathOperator{\CC}{\mathbb{C}}
\DeclareMathOperator{\hg}{\mathfrak{h}_g}

\newcommand{\thchar}[2] {\begin{bmatrix}#1\\#2\end{bmatrix}}
\newcommand{\thcharsm}[2] {\left[ \begin{smallmatrix} #1
      \\ #2 \end{smallmatrix} \right]}


\title{Efficient Modular Computation of the Riemann Theta Function}
\author{
  \small
  \begin{tabular}{cc}
    Chris Swierczewski & Grady Williams \\
    University of Washington & University of Washington \\
    Department of Applied Mathematics & Department of Mathematics \\
    Seattle, WA 98195-3925 & Seattle, WA 98195-4350 \\
    \href{mailto:cswiercz@uw.edu}{\tt cswiercz@uw.edu} &
    \href{mailto:gradyrw@uw.edu}{\tt gradyrw@uw.edu}
  \end{tabular}
}
\date{\today}

%%%%%%%%%%%%%%%%%%%%%%%%%%%%%%%%%%%%%%%%%%%%%%%%%%%%%%%%%%%%%%%%%%%%%%%%%%%%%%%
\begin{document}
%%%%%%%%%%%%%%%%%%%%%%%%%%%%%%%%%%%%%%%%%%%%%%%%%%%%%%%%%%%%%%%%%%%%%%%%%%%%%%%

\maketitle

\begin{abstract}
We present an algorithm for uniformly approximating the Riemann theta function
$\theta : \CC^g \times \hg \to \CC$ with repsect to both of its inputs; a
complex vector $z \in \CC^g$ and a Riemann matrix $\Omega \in \hg$. This
approximation allows for efficient parallel computation of $\theta(z,\Omega)$
for varying $z$ and $\Omega$. The algorithm is implemented in the Python, C,
and CUDA programming languages and is a part of the {\sc abelfunctions}
software package. \url{http://github.com/cswiercz/abelfunctions}
\end{abstract}


%%%%%%%%%%%%%%%%%%%%%%%%%%%%%%%%%%%%%%%%%%%%%%%%%%%%%%%%%%%%%%%%%%%%%%%%%%%%%%%
\section{Introduction}\label{sec:intro}
%%%%%%%%%%%%%%%%%%%%%%%%%%%%%%%%%%%%%%%%%%%%%%%%%%%%%%%%%%%%%%%%%%%%%%%%%%%%%%%

The Riemann theta function is the fundamental building for Abelian
functions. These are functions in $g$ complex variables, which are meromorphic
and double periodic in each variable. As such, the Abelian functions generalize
the classical elliptic functions to multiple variables. Like elliptic
functions, Abelian functions arise in a plethora of applications and the
Riemann theta function itself sees applications in fields such as integrable
systems of partial differential equations, convex optimization, and number
theory.

We address the problem of efficiently computing the Riemann theta function to a
desired floating point precision. In particular, we derive methods for
uniformly approximating the Riemann theta function with respect to its input
with the goal of providing an efficient method of calculation.

%%%%%%%%%%%%%%%%%%%%%%%%%%%%%%%%%%%%%%%%%%%%%%%%%%%%%%%%%%%%%%%%%%%%%%%%%%%%%%%
\section{Riemann Theta Functions}\label{sec:riemanntheta}
%%%%%%%%%%%%%%%%%%%%%%%%%%%%%%%%%%%%%%%%%%%%%%%%%%%%%%%%%%%%%%%%%%%%%%%%%%%%%%%

The Riemann theta function, $\theta : \CC^g \times \hg \to \CC$, is defined in
terms of its Fourier series
\begin{equation}\label{eq:theta}
  \theta(z,\Omega) = \sum_{n \in \mathbb{Z}^g}
  e^{2\pi i \left( \tfrac{1}{2} n \cdot \Omega n + n \cdot z \right)}
\end{equation}
where $\hg$ is the $g$-dimensional Siegel upper half space; that is, the space
of all matrices $\Omega \in \CC^{g \times g}$ such that $\Omega^T = \Omega$ and
$\text{Im}(\Omega)$ is positive definite. The Riemann theta function converges
absolutely and uniformly on compact sets in $\CC^g \times \hg$. It is periodic
in $z$ with integer periods and quasi-periodic in $z$ in the columns of
$\Omega$. That is, if $m,n \in \ZZ^g$ then
\begin{equation}\label{eq:theta transform}
  \theta(z + \Omega m + n, \Omega) = e^{-2\pi i \left( \tfrac{1}{2} m
    \cdot \Omega m + m \cdot z \right)} \theta(z,\Omega).
\end{equation}
See \cite{DLMF,Mumford:2007ww,Mumford:2007wm} for further properties of the
Riemann theta funciton.

%------------------------------------------------------------------------------
\subsection{Uniform Approximation with Respect to $z$}
%------------------------------------------------------------------------------

From the definition of the Riemann theta function and its quasi-periodicity
property we immediately see two challenges in computing $\theta(z,\Omega)$ up
to a desired floating point precision. First, the function is given in terms of
an infinite sum. Second, theta grows doubly exponentially with respect to $z$
in $g$ linearly independent directions; the issue here being about how to
handle any floating point calculation errors that surface.

[...]

We now summarize a previous result due to Deconinck, Heil, Bobenko, van Hoeij,
and Schmies allows us to approximate $\theta(z,\Omega)$ for a fixed $\Omega \in
\hg$ and any choice of $z \in \CC^g$ to a desired precision $\epsilon$
\cite{Deconinck:2004uc}. In the sequel let $z=x + iy, \, \Omega = X + iY, \, Y
= T^TT$ be the Cholesky decomposition of $Y$, $\Gamma(s,x)$ be the incomplete
gamma function, and given a vector $w \in \RR^g$ we let $[w]$ be the vector
with integer components closest to $v$ and $[[w]] = w - [w]$.

\begin{theorem}[Uniform Approximation with Respect to $z$.]
The Riemann theta function is approximated by
\begin{equation} \label{eqn: uniform}
  \theta(z,\Omega)
  \approx
  e^{\pi y \cdot Y^{-1} y}
  \sum_{U_R} e^{2 \pi i \left( \tfrac{1}{2} \left( n - [Y^{-1}y] \right)
    \cdot X \left( n - [Y^{-1}y] \right) +
    \left( n - [Y^{-1}y] \right) \cdot x \right)}
    e^{- ||v(n)||^2}
\end{equation}
with absolute error $\epsilon$. The approximate of the sum is uniform in $z$.
Here $v(n) = \sqrt{\pi} T \left( n + [[Y^{-1}y]] \right)$,
\begin{equation} \label{eqn: uniform points}
  U_R = \left\{ n \in \ZZ^g \; | \; \pi (n-c) \cdot Y (n-c) < R^2,
                |c_j| < 1/2 \; \forall j = 1,\ldots,g \right\}.
\end{equation}
Let $\Lambda = \left\{ \sqrt{\pi}T \left( n + [[Y^{-1}y]] \right) \; | \; n \in
\ZZ^g \right\}$. The shortest distance between any two points of $\Lambda$ is
denoted by $\rho$. Then the radius $R$ is determined as the greater of
$(\sqrt{2g} + \rho)/2$ and the real positive solution of $\epsilon =
g2^{g-1}\Gamma(g/2,(R-\rho/2)^2)/\rho^g$.
\end{theorem}


Note that $\rho$ is also equal to the length of the shorest lattice vector of
the lattice $\hat{\Lambda} = \sqrt{\pi}T\ZZ^g$.


%%%%%%%%%%%%%%%%%%%%%%%%%%%%%%%%%%%%%%%%%%%%%%%%%%%%%%%%%%%%%%%%%%%%%%%%%%%%%%%
\section{Siegel Transformation and Sympectic Geometry}\label{sec:siegel}
%%%%%%%%%%%%%%%%%%%%%%%%%%%%%%%%%%%%%%%%%%%%%%%%%%%%%%%%%%%%%%%%%%%%%%%%%%%%%%%



In the previous section, the subset of $\ZZ^g$ over which we truncate our
evaluation of theta, $U_R$, is a collection of integer vectors lying inside the
ellipsoid $\pi z \cdot Yz = R$. If the ellipsoid has high eccentricity then it
is possible that both $|U_R|$ is large and the magnitudes of the elements of
$U_R$ are large. These two situations can contribute to numerical error in an
implementation of the algorthm.




%------------------------------------------------------------------------------
\subsection{Theta Functions Under Modular Transformations}
%------------------------------------------------------------------------------

This high eccentricity situation can be mitigated by using modular
transformations on the Riemann matirx $\Omega$ in order to map the problem into
one where we instead compute $\theta(\hat{z}, \hat{\Omega})$ on a new lattice
$\hat{U}_R$ that has smaller eccentricity.

Given a symplectic matrix over the integers
\[
\Gamma = \left( \begin{matrix} a & b \\ c & d \end{matrix} \right)
\in SP(2g,\mathbb{Z})
\]
$\Gamma$ acts on $\hg$ by the ``modular transformation''
\[
\Omega \mapsto \hat{\Omega} = (a\Omega + b) (c\Omega + d)^{-1}.
\]
By [Tata I], the Riemann theta function satisfies the following functional
equations under certain modular transformations. The listed modular
tranformations form a basis for the modular group $SP(2g, \mathbb{Z})$.
\begin{gather} \label{eqn: full transformation}
\theta \left( 
(c \Omega + d)^{-1} z, (a \Omega + b)(c \Omega + d)^{-1}
\right) = \notag \\
\qquad \qquad
\sqrt{\det(c \Omega + d)}
e^{\pi i z \cdot (c \Omega + d)^{-1} c z}
\theta(z, \Omega)
\end{gather}


\begin{table}[h]
\label{tbl: transformations}

\centering
\begin{tabular}{ccc}
  Symplectic Transform & Transformed Matrix & Functional Eq. \\ \hline
  $\Gamma = \left[ \begin{smallmatrix} a & 0 \\ 0 &
      a^{-T} \end{smallmatrix} \right]$ & $\hat{\Omega} = a \Omega
  a^T$ & $\theta(z,\Omega) = \theta(az,a\Omega a^T)$ \\

  $\Gamma = \left[ \begin{smallmatrix} I & b \\ 0 &
      I \end{smallmatrix} \right]$ & $\hat{\Omega} = \Omega + b$ &
  $\theta(z,\Omega) = \theta(z, \Omega + b)$ \\

  $\Gamma = \left[ \begin{smallmatrix} 0 & -I \\ I &
      0 \end{smallmatrix} \right]$ & $\hat{\Omega} = -\Omega^{-1}$ &
  $\theta(\Omega^{-1}, -\Omega^{-1}) =$ \\

  & & $\sqrt{\det(-i\Omega)}e^{\pi i z \cdot \Omega^{-1} z} \theta(z,\Omega)$
\end{tabular}
\end{table}



\section{Algorithm}

\begin{algorithm}
\caption{Siegel Transform of a Riemann Matrix}
\label{alg: siegel}
\begin{algorithmic}
  \Procedure{Siegel}{$\Omega$} 
  \Comment{$\Omega$ a $g \times g$ Riemann matrix}

  \State $\Gamma \gets I_{2g}$
  \Comment{$I_{2g} = 2g \times 2g$ identity matrix}
  \State $a \gets \left[ \begin{smallmatrix} 0 & \mathbf{0}^T \\
      \mathbf{0} & I_{g-1} \end{smallmatrix} \right]$
  \Comment{$\mathbf{0} = $ column vector with $g-1$ zeros}
  \State $b \gets \left[ \begin{smallmatrix} -1 & \mathbf{0}^T \\
      \mathbf{0} & \mathbf{0}_{g-1} \end{smallmatrix} \right]$
  \Comment{$\mathbf{0}_{g-1} = (g-1) \times (g-1)$ zero matrix}
  \State $c \gets \left[ \begin{smallmatrix} 1 & \mathbf{0}^T \\
      \mathbf{0} & \mathbf{0}_{g-1} \end{smallmatrix} \right]$
  \State $d \gets \left[ \begin{smallmatrix} 0 & \mathbf{0}^T \\
      \mathbf{0} & I_{g-1} \end{smallmatrix} \right]$
  \State $\texttt{finished} \gets \texttt{False}$
  \While{{\tt finished} $ \neq $ {\tt True}}
    \State $Y \gets \text{Im } \Omega$
    \State $T \gets \textsc{Cholesky}(Y)^T$
    \State $L \gets \textsc{SortedLLL}(T)$
    \Comment{LLL with cols. sorted by length}
    \State $U \gets$ solution to $TU = L$
    \State $\Omega \gets U^T \Omega U, \quad \Gamma \gets \left[
      \begin{smallmatrix} U^T & 0 \\ 0 & U^{-1} \end{smallmatrix}
      \right] \Gamma$
    \State $X \gets \text{Re } \Omega$
    \State $\Omega \gets \Omega - [X], \quad \Gamma \gets \left[
      \begin{smallmatrix} I & -[X] \\ 0 & I \end{smallmatrix}
      \right] \Gamma$
    \If{$|\Omega_{11}| \geq 1$}
      \State $\texttt{finished} \gets \texttt{True}$
    \Else
      \State $\Gamma \gets \left[
        \begin{smallmatrix} a & b \\ c & d \end{smallmatrix}
        \right] \Gamma$
      \State $\Omega \gets (a\Omega + b)(c \Omega + d)^{-1}$
      \Comment{Compute inverse using stable solver.}
    \EndIf
  \EndWhile
  \State \textbf{return} $\Omega, \Gamma$
\EndProcedure
\end{algorithmic}
\end{algorithm}


%%%%%%%%%%%%%%%%%%%%%%%%%%%%%%%%%%%%%%%%%%%%%%%%%%%%%%%%%%%%%%%%%%%%%%%%%%%%%%%
\section{Efficient Computation of Theta}\label{sec:computation}
%%%%%%%%%%%%%%%%%%%%%%%%%%%%%%%%%%%%%%%%%%%%%%%%%%%%%%%%%%%%%%%%%%%%%%%%%%%%%%%



\bibliographystyle{plain}
\bibliography{../papers.bib}

\end{document}

